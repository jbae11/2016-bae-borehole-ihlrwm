\documentclass{anstrans}
%%%%%%%%%%%%%%%%%%%%%%%%%%%%%%%%%%%
\title{Benefits of Siting a Borehole Repository at a Non-Operating Nuclear 
Facility}
\author{Jin Whan Bae, William Roy, Kathryn Huff}

\institute{
Dept. of Nuclear Plasma, and Radiological Engineering, University of Illinois at Urbana-Champaign
\and
Urbana, IL
}

\email{jbae11@illinois.edu}

%%%% packages and definitions (optional)
\usepackage{graphicx} % allows inclusion of graphics
\usepackage{booktabs} % nice rules (thick lines) for tables
\usepackage{microtype} % improves typography for PDF

\newcommand{\SN}{S$_N$}
\renewcommand{\vec}[1]{\bm{#1}} %vector is bold italic
\newcommand{\vd}{\bm{\cdot}} % slightly bold vector dot
\newcommand{\grad}{\vec{\nabla}} % gradient
\newcommand{\ud}{\mathop{}\!\mathrm{d}} % upright derivative symbol

%%%% Acronym support

\usepackage[acronym,toc]{glossaries}
\include{acros}
\makeglossaries

\begin{document}
%%%%%%%%%%%%%%%%%%%%%%%%%%%%%%%%%%%%%%%%%%%%%%%%%%%%%%%%%%%%%%%%%%%%%%%%%%%%%%%%
\section{Introduction}

This work evaluates a potential solution for two pressing matters in the viability of
nuclear energy: spent fuel disposal and power plants that no longer operate. The 
potential benefits of siting a borehole repository at a shut down nuclear power 
plant facility are analyzed from the perspective of myriad stakeholders. 
Preliminary results indicate that integrated siting will make economic 
use of the shut down power plant, take advantage of spent fuel handling 
infrastructure at those sites, help to empty the crowded spent 
fuel storage pools at nearby reactors, and will do so at sites more likely to 
have consenting communities.


%%%%%%%%%%%%%%%%%%%%%%%%%%%%%%%%%%%%%%%%%%%%%%%%%%%%%%%%%%%%%%%%%%%%%%%%%%%%%%%%

\subsection{Motivation}
The proposed integrated siting strategy takes advantage of three technical 
benefits of borehole repository designs: modularity, broad geologic 
suitability, and footprint efficiency. Modularity enables regional repositories 
to scale in size according to the local spent fuel burden. 
Additionally, the necessary geological characteristics required for borehole 
disposal, crystalline basement rocks at $2,000m - 5,000m$ deep, are relatively 
common in stable continental regions \cite{arnold_research_2012}. Finally, the 
surface footprint requirements of a borehole repository are comparable to the 
available footprint of a nuclear power reactor site, with only $30km^2$ 
required for the total \gls{SNF} amount proposed for Yucca Mountain 
\cite{brady_deep_2009}.

Integrated siting also has potential economic benefits. One 
significant cost inherent to borehole repository concepts is the repacking of 
spent fuel assemblies into smaller-diameter waste canisters. However, siting a 
repository at a non-operating power plant facility, especially one with a 
dry-cask storage site, will take advantage of already existing infrastructure 
and local human talent for spent fuel handling and packaging. Many candidate 
non-operating reactor sites, such as those mapped in Figure \ref{fig:shutdown} 
may be appropriate for integrated siting if they are located above crystalline 
basement formations and include dry cask packaging facilities.


Preliminary work \cite{waleed_regional_2015} indicates integrated siting is 
appealing to many stakeholder groups. For example, a consent-based approval 
process may be feasible since communities local to power plants may be uniquely 
receptive to the incentives of hosting a repository.  

%%%%%%%%%%%%%%%%%%%%%%%%%%%%%%%%%%%%%%%%%%%%%%%%%%%%%%%%%%%%%%%%%%%%%%%%%%%%%%%

\section{Methodology}

This work will evaluate the potential impacts of integrated siting from the 
perspective of 5 stakeholders:
\begin{itemize}
        \item the federal government,
        \item the state government,
        \item the local government,
        \item the local community,
        \item and the owner of the non-operating plant.
\end{itemize}


Preliminary work \cite{waleed_regional_2015} suggests that integrated siting 
will reduce costs, construction, time (both for construction and licensing), 
transportation distances, and resistence from the local community.  The present 
work will compare the proposal along these axes to a base case: a standalone 
borehole repository at a similar location to that of Yucca Mountain.  
Quantification of those stakeholder benefits will be undertaken for two 
different regions of the US in addition to the base case.  


\begin{figure}[!h] 
  \centering
  \includegraphics[width=0.8\columnwidth]{power-reactors-decommissioning}	
  \caption{Non-operating facilities status
  \cite{nuclear_regulatory_commission_nrc_2015}.}
  \label{fig:shutdown}
\end{figure}

\section{Acknowledgments}

This material is based upon work supported by \gls{ACDIS}. Preliminary work was 
conducted in collaboration with the \gls{ISRG} within the \gls{NPRE}. The 
authors are accordingly grateful for guidance by Prof. Clifford Singer.

%%%%%%%%%%%%%%%%%%%%%%%%%%%%%%%%%%%%%%%%%%%%%%%%%%%%%%%%%%%%%%%%%%%%%%%%%%%%%%%%
\bibliographystyle{ans}
\bibliography{bibliography}
\end{document}
