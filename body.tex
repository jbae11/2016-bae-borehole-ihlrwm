\documentclass{anstrans}

%%%% packages and definitions (optional)
\usepackage{graphicx} % allows inclusion of graphics
\usepackage{booktabs} % nice rules (thick lines) for tables
\usepackage{microtype} % improves typography for PDF

\newcommand{\SN}{S$_N$}
\renewcommand{\vec}[1]{\bm{#1}} %vector is bold italic
\newcommand{\vd}{\bm{\cdot}} % slightly bold vector dot
\newcommand{\grad}{\vec{\nabla}} % gradient
\newcommand{\ud}{\mathop{}\!\mathrm{d}} % upright derivative symbol


\title{Benefits of Siting a Borehole Repository on Non-Operating Nuclear 
Facility}
\author{Jin Whan Bae, William Roy, Kathryn Huff}

\institute{
Dept. of Nuclear, Plasma, and Radiological Engineering, University of Illinois at Urbana-Champaign
\and
Urbana, IL
}
\email{jbae11@illinois.edu}


%%%% Acronym support

\usepackage[acronym,toc]{glossaries}
\include{acros}

	
\makeglossaries




\begin{document}

%%%%%%%%%%%%%%%%%%%%%%%%%%%%%%%%%%%%%%%%%%%%%%%%%%%%%%%%%%%%
\section{Case Definition and Methodology}

A proposed case is building a 70,000 \gls{MTHM} capacity borehole repository at the Clinton Power Plant in Illinois. The base case is to build a standalone borehole repository at a location similar to that of Yucca Mountain with the same capacity. 

\subsection{Proposed Case Methodology and Definition}
The proposed case is siting a borehole repository at a shut-down power plant.
 In order to minimize transport cost, a central location is preferred. An
  elementary analysis on the transportation of spent fuel is done by
   calculating the total amount of waste times the distance it has to travel
    ( in units of MTHM*km). The distance between each storage site 
    (i.e. reactors and \gls{ISFSI}) is calculated by 
    using the havershine formula on the geographical coordinates of the sites. The
     coordinates and spent fuel inventory data
    is from the GC-859 data from the \gls{EIA} %\cite{?????}
    and the \gls{CURIE} website.
     From the list of 74 reactors, several candidates
    with the smallest MTHM*Km value is listed below:
    
    
\begin{table}[h]
\centering
    \caption { Reactors with least MTHM*Km value}
	\begin{tabular}{l|l|l|l}
	\hline
	Reactor & State & $MTHM*km$ & License Area [$km^2$]  \\ \hline
	Clinton & Illinois &  77,352,339 & 57.87   \\ \hline
	Peach Bottom & Pennsylvania & 85,563,135 & 2.509   \\ \hline
	Indian Point &   New York & 77,352,339 & ?????   \\ \hline
	Dresden & Illinois &  77,663,969 & 3.856   \\ \hline
	
	\end{tabular}
\end {table}


The Clinton Power Plant is chosen as the site for the proposed case due to its
low $MTHM*km^2$ value and substantially large license area. Considering that only
 $30km^2$ is required for all the total \gls{SNF} amount, the licensed area at Clinton power plant allows more than  enough space to site a borehole repository, which avoids
  possible conflicts with the community from purchasing and utilizing more land.
  
\subsection{Base Case Methodology and Definition}
The base case is presented in order to demonstrate the cost savings and efficiencies 
that arise from the proposed case. The base case mimics the Yucca Mountain Project
 except the design. Costs include new licensing and processing facility for repacking 
the spent fuel assemblies.

%%%%%%%%%%%%%%% Temporarily disabled Table %%%%%%%%%%%%%%%%%%%%%%%%%%%%%%%
    
\iffalse

\begin{table}[h]
\centering
\caption {Incentive Criterion and Weight for Each Stakeholder}
	\begin{tabular}{l|l|l|l|l|l}
	\hline
	 & Federal & State & Local & Utility & Environmental \\ \hline
	Job Creation &   & 1 & 3 & 1 &   \\ \hline
	Transport[$MTHM*km$] & 2 & 1 & 2 & & 2\\ \hline
	No Need for new treatment license & 2 & & & 1 & \\ \hline
	Emptying Spent Fuel Storage Pools & 3 & & & 3 & \\ \hline
	Net Cost & 3 & & & 3 & \\ \hline
	No New Above-Ground Facility Construction & 3 & & & 3 & \\ \hline
	
	\end{tabular}
\end{table}

\fi



%%%%%%%%%%%%%%% Temporarily disabled Table %%%%%%%%%%%%%%%%%%%%%%%%%%%%%%%

\bibliographystyle{ans}
\bibliography{bibliography}

\end{document}