
%%%%%%%%%%%%%%%%%%%%%%%%%%%%%%%%%%%%%%%%%%%%%%%%%%%%%%%%%%%%%%%%%%%%%%%%%%%%%%%%
\section{Introduction}
Creative solutions are necessary to meet the \gls{SNF} disposal challenges 
faced by the United States.
This work proposes and evaluates a strategy that leverages the 
remaining resources inherent in a shut down nuclear reactor site toward a new 
purpose:a spent fuel repository facility.

Domestic nuclear power plants are at risk of shutdown in areas with surplus 
electricity capacity from coal and natural gas. Kewaunee and Crystal River 
have already closed  and numerous other plants are at risk in the near term 
\cite{nei_nuclear_2016}.  Simultaneously, the \gls{DOE} 
has begun to move forward with consent-based siting of a nuclear 
spent fuel repository \cite{doe_designing_2016}. The proposed solution in this 
work seeks to combine these efforts toward a more economic and politically 
feasible solution. 

% explain that this work compares the base case with the proposal
This work considers the potential benefits of siting a borehole-type repository 
at the site of a shut-down nuclear power plant.  The  expected benefits of this 
proposed integrated siting strategy include reduced radioactive waste 
transportation burden, increased likelihood of consent from the local 
community, and improved expediency achieved through leveraging existing 
infrastructure and skill.

% note the metrics on which these two cases are being compared.
The siting strategy will be compared to a reference case at 
Yucca Mountain through quantitative metrics. The incentives of various 
stakeholders will also be modeled as a weighted linear sum of these 
metrics. 

\subsection{Motivation}
The proposed integrated siting strategy takes advantage of three technical 
benefits of borehole repository designs: modularity, broad geological 
suitability, and footprint efficiency. Modularity enables regional repositories 
to scale in size according to the local spent fuel burden. 
Additionally, the necessary geological characteristics required for borehole 
disposal, crystalline basement rocks at $2,000 m - 5,000 m$ deep, are relatively 
common in stable continental regions \cite{arnold_research_2012}. Finally, the 
surface footprint requirements of a borehole repository are comparable to the 
available footprint of a nuclear power reactor site, with only $30 km^2$ 
required for the total \gls{SNF} amount proposed for Yucca Mountain 
\cite{brady_deep_2009}.

Integrated siting also has potential economic benefits. One significant cost 
inherent to borehole repository concepts is the repacking of spent fuel 
assemblies into smaller-diameter waste canisters representing over 15\% of 
estimated per-borehole cost \cite{arnold_reference_2011}.  However, siting a 
repository at a non-operating power plant facility, especially one with a 
dry-cask storage site, will take advantage of already existing infrastructure 
and local human talent for spent fuel handling and packaging. Many candidate 
non-operating reactor sites, such as those mapped in Figure \ref{fig:shutdown}, 
may be appropriate for integrated siting if they are located above crystalline 
basement formations and include dry cask packaging facilities.

\begin{figure}[htpb!] 
  \centering
  \includegraphics[width=0.8\columnwidth]{power-reactors-decommissioning}	
  \caption{Non-operating facilities status
  \cite{nuclear_regulatory_commission_nrc_2015}.}
  \label{fig:shutdown}
\end{figure}

Finally, integrated siting may be more practically and politically feasible. 
Preliminary work \cite{waleed_regional_2015} indicates integrated siting is 
appealing to many stakeholder groups. For example, a consent-based approval 
process may be feasible because communities local to power plants may be 
uniquely receptive to the incentives of hosting a repository. This paper seeks to
 quantify the impact of these and other features of the proposed siting strategy. 

%%%%%%%%%%%%%%%%%%%%%%%%%%%%%%%%%%%%%%%%%%%%%%%%%%%%%%%%%%%%%%%%%%%%%%%%%%%%%%%


